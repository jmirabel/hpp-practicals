% a. présentation du cadre
%   - cas d'usage
%   - quelques définitions
%   - quelques fonctions exemples (position, orientation, COM)
% b. pour les configurations
%   - méthode exacte.
%   - méthode iterative (Newton-Raphson, conditions KKT pour les limites articulaires).
%   - comparaisons des méthodes.
% c. pour les trajectoires (seulement mettre en évidence le problème)

\begin{frame} {Motivations}
  \begin{itemize}
    \item Comment g\'en\'erer une configuration satisfaisant des crit\`eres g\'eom\'etriques ?
      \begin{itemize}
        \item position,
        \item orientation,
        \item centre de masse,
        \item visibilit\'e,
        \item \ldots
      \end{itemize}
      \pause
    \item Interpolation lin\'eraire dans $\CS$:
      \begin{itemize}
        \item courbe non lin\'eraire dans $\WS$
      \end{itemize}
      \pause
  \end{itemize}
\end{frame}

\begin{frame}[<+->]{D\'efinition}
  \emph{Contrainte} : une equation de la forme $$f(q) = 0$$
  \begin{itemize}
    \item<2-> $f \in \mathcal{C}^1(\CS, \real^m)$,
    \item<3-> $m$ est la dimension de la contrainte.
  \end{itemize}

  \onslide+<4->{
    Quelques exemples :
    \begin{itemize}
      \item<5-> position,
      \item<6-> centre de masse,
      \item<7-> orientation.
    \end{itemize}
  }
\end{frame}

\begin{frame}[<+->]{R\'esolution exacte}
  \begin{itemize}
    \item donne l'ensemble des solutions d'une contrainte : ${ q\in\CS, f(q) = 0}$,
    \item possible pour certaines contraintes mais :
      \begin{itemize}
        \item sp\'ecifique \`a chaque cas,
        \item complexes \`a impl\'ementer,
        \item difficile, voire impossible \`a combiner.
      \end{itemize}
  \end{itemize}
\end{frame}

\begin{frame}[<+->]{R\'esolution num\'erique}
  \emph{M\'ethode it\'erative}
  \begin{itemize}
    \item si $ || f(\conf_n) || < \epsilon $ alors $\conf_n$ est une solution,
    \item calcul de la d\'eriv\'ee de $f$ :
      $$J = \frac{\partial f}{\partial \conf}$$
    \item calcul de $\conf_{n+1}$ avec une approximation
      \begin{itemize}
        \item du 1er ordre
          $$\conf_{n+1} = \conf_n - \alpha \left( \frac{||f(\conf_n)||^2}{2 ||J^T f(\conf_n)||^2} \right) J^{T} f(\conf_n), \alpha\in]0,1] $$
        \item du 2nd ordre: $J^{\dagger}$ est la pseudo-inverse de $J$
          $$\conf_{n+1} = \conf_n - \alpha J^{\dagger} f(\conf_n), \alpha\in]0,1] $$
      \end{itemize}
  \end{itemize}
\end{frame}

\begin{frame}[<+->]{Digression: pseudo-inverse}
  Soit $J\in\mathcal{M}_{m,n}(\real), n \ge m$ et soit le probl\`eme 
  $J x = b$.

  \begin{itemize}
    \item $x^* = J^{\dagger} b$ est la solution de norme minimale.
    \item la matrice $I_n - J^\dagger J$ est un projecteur sur le noyau de $J$.
    \item l'ensemble des solutions est $\left\{ x^* + (I_n - J^\dagger J) u, u\in\real^n \right\}$.
  \end{itemize}
\end{frame}

%\begin{frame}[<+->]{Digression: espace null (noyau) d'une matrice}
%  Soit $J\in\mathcal{M}_{n,m}(\real), n \ge m$.
%  La pseudo-inverse $J^{\dagger}$ de $J$ est l'unique matrice qui v\'erifie.
%  alors $J$
%%utilisation de la d\'ecomposition en valeur singuli\`ere (SVD):
%\end{frame}
