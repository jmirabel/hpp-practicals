%\section {Random methods}

%
%  histoire
%

\begin{frame}[<+->]{Un peu d'histoire}
  Les premi\`eres approches sont d\'eterministes :
  \begin{itemize}
    \item discretisation,
      \begin{itemize}
        \item dimensionnalit\'e
      \end{itemize}
    \item diagrammes de Vorono\"i
      \begin{itemize}
        \item dur \`a g\'en\'eraliser pour plus de 2-3 dimensions.
        \item dimensionnalit\'e
      \end{itemize}
    \item décomposition cellulaire
      \begin{itemize}
        \item dur \`a g\'en\'eraliser pour plus de 2-3 dimensions.
        \item dimensionnalit\'e
      \end{itemize}
    \item champs de potentiel
      \begin{itemize}
        \item dur \`a g\'en\'eraliser pour des corps complexes,
        \item sujet au probl\`eme des minimums locaux.
      \end{itemize}
  \end{itemize}
  \onslide+<+->{\'Emergence de m\'ethodes al\'eatoires dans les ann\'ees 1990.}
\end{frame}

%
%  random methods
%

\begin{frame} {M\'ethodes al\'eatoires}
  Principe :
  \begin{itemize}
    \item tirer une configuration al\'eatoire,
      \pause
    \item construire un graphe (carte) dont les noeuds sont des configurations,
      \pause
    \item et dont les ar\^etes sont des interpolations lin\'eaires sans collision.
  \end{itemize}
\end{frame}

%
%  La méthode du réseau aléatoire
%

\begin{frame} {Probabilistic roadmap (PRM) 1994}
\centerline {
  \includegraphics[width=.8\linewidth]{figures/PRM1.pdf}
}
\end{frame}

\begin{frame} {Probabilistic roadmap (PRM) 1994}
\centerline {
  \includegraphics[width=.8\linewidth]{figures/PRM2.pdf}
}
\end{frame}

\begin{frame} {Probabilistic roadmap (PRM) 1994}
\centerline {
  \includegraphics[width=.8\linewidth]{figures/PRM3.pdf}
}
\end{frame}

\begin{frame} {Probabilistic roadmap (PRM) 1994}
\centerline {
  \includegraphics[width=.8\linewidth]{figures/PRM4.pdf}
}
\end{frame}

\begin{frame} {Probabilistic roadmap (PRM) 1994}
\centerline {
  \includegraphics[width=.8\linewidth]{figures/PRM5.pdf}
}
\end{frame}

\begin{frame} {Probabilistic roadmap (PRM) 1994}
\centerline {
  \includegraphics[width=.8\linewidth]{figures/PRM6.pdf}
}
\end{frame}

\begin{frame} {Probabilistic roadmap (PRM) 1994}
\centerline {
  \includegraphics[width=.8\linewidth]{figures/PRM7.pdf}
}
\end{frame}

\begin{frame} {Probabilistic roadmap (PRM) 1994}
\centerline {
  \includegraphics[width=.8\linewidth]{figures/PRM8.pdf}
}
\end{frame}

\begin{frame} {Probabilistic roadmap (PRM) 1994}
\centerline {
  \includegraphics[width=.8\linewidth]{figures/PRM9.pdf}
}
\end{frame}

\begin{frame} {Probabilistic roadmap (PRM) 1994}
\centerline {
  \includegraphics[width=.8\linewidth]{figures/PRM10.pdf}
}
\end{frame}

\begin{frame} {Probabilistic roadmap (PRM) 1994}
\centerline {
  \includegraphics[width=.8\linewidth]{figures/PRM11.pdf}
}
\end{frame}

\begin{frame} {Probabilistic roadmap (PRM) 1994}
\centerline {
  \includegraphics[width=.8\linewidth]{figures/PRM12.pdf}
}
\end{frame}

\begin{frame} {Probabilistic roadmap (PRM) 1994}
\centerline {
  \includegraphics[width=.8\linewidth]{figures/PRM13.pdf}
}
\end{frame}

\begin{frame} {Probabilistic roadmap (PRM) 1994}
\centerline {
  \includegraphics[width=.8\linewidth]{figures/PRM14.pdf}
}
\end{frame}

\begin{frame} {Probabilistic roadmap (PRM) 1994}
\centerline {
  \includegraphics[width=.8\linewidth]{figures/PRM15.pdf}
}
\end{frame}

\begin{frame} {Probabilistic roadmap (PRM) 1994}
\centerline {
  \includegraphics[width=.8\linewidth]{figures/PRM16.pdf}
}
\end{frame}

\begin{frame} {Probabilistic roadmap (PRM) 1994}
\centerline {
  \includegraphics[width=.8\linewidth]{figures/PRM17.pdf}
}
\end{frame}

\begin{frame} {Probabilistic roadmap (PRM) 1994}
\centerline {
  \includegraphics[width=.8\linewidth]{figures/PRM18.pdf}
}
\end{frame}

\begin{frame} {Probabilistic roadmap (PRM)}
  \begin{itemize}
  \item Beaucoup de noeuds inutiles sont cr\'e\'es,
    \begin{itemize}
    \item cela augmente le co\^ut de connexion de nouveaux noeuds \'a la carte courrante.
    \end{itemize}
  \item Am\'elioration: Visibility-based PRM
    \begin{itemize}
    \item Seul les noeuds \textit{int\'eressants} sont gard\'es.
    \end{itemize}
  \end{itemize}
\end{frame}


%
%  Visibility-based probabilistic roadmap
%
\begin{frame} {Visibility-based probabilistic roadmap (Visi-PRM) 1999}
\centerline {
  \includegraphics[width=.8\linewidth]{figures/VPRM1.pdf}
}
\end{frame}

\begin{frame} {Visibility-based probabilistic roadmap (Visi-PRM) 1999}
\centerline {
  \includegraphics[width=.8\linewidth]{figures/VPRM2.pdf}
}
\end{frame}

\begin{frame} {Visibility-based probabilistic roadmap (Visi-PRM) 1999}
\centerline {
  \includegraphics[width=.8\linewidth]{figures/VPRM3.pdf}
}
\end{frame}

\begin{frame} {Visibility-based probabilistic roadmap (Visi-PRM) 1999}
\centerline {
  \includegraphics[width=.8\linewidth]{figures/VPRM4.pdf}
}
\end{frame}

\begin{frame} {Visibility-based probabilistic roadmap (Visi-PRM) 1999}
\centerline {
  \includegraphics[width=.8\linewidth]{figures/VPRM5.pdf}
}
\end{frame}

\begin{frame} {Visibility-based probabilistic roadmap (Visi-PRM) 1999}
\centerline {
  \includegraphics[width=.8\linewidth]{figures/VPRM6.pdf}
}
\end{frame}

\begin{frame} {Visibility-based probabilistic roadmap (Visi-PRM) 1999}
\centerline {
  \includegraphics[width=.8\linewidth]{figures/VPRM7.pdf}
}
\end{frame}

\begin{frame} {Visibility-based probabilistic roadmap (Visi-PRM) 1999}
\centerline {
  \includegraphics[width=.8\linewidth]{figures/VPRM8.pdf}
}
\end{frame}

\begin{frame} {Visibility-based probabilistic roadmap (Visi-PRM) 1999}
\centerline {
  \includegraphics[width=.8\linewidth]{figures/VPRM9.pdf}
}
\end{frame}

\begin{frame} {Visibility-based probabilistic roadmap (Visi-PRM) 1999}
\centerline {
  \includegraphics[width=.8\linewidth]{figures/VPRM10.pdf}
}
\end{frame}

\begin{frame} {Visibility-based probabilistic roadmap (Visi-PRM) 1999}
\centerline {
  \includegraphics[width=.8\linewidth]{figures/VPRM11.pdf}
}
\end{frame}

\begin{frame} {Visibility-based probabilistic roadmap (Visi-PRM) 1999}
\centerline {
  \includegraphics[width=.8\linewidth]{figures/VPRM12.pdf}
}
\end{frame}

\begin{frame} {Visibility-based probabilistic roadmap (Visi-PRM) 1999}
\centerline {
  \includegraphics[width=.8\linewidth]{figures/VPRM13.pdf}
}
\end{frame}

\begin{frame} {Visibility-based probabilistic roadmap (Visi-PRM) 1999}
\centerline {
  \includegraphics[width=.8\linewidth]{figures/VPRM14.pdf}
}
\end{frame}

\begin{frame} {Visibility-based probabilistic roadmap (Visi-PRM) 1999}
\centerline {
  \includegraphics[width=.8\linewidth]{figures/VPRM15.pdf}
}
\end{frame}

%
%  La méthode des arbres aléatoires d'exploration rapide.
%

\begin{frame} {Rapidly exploring Random Tree (RRT) 2000}
\centerline {
  \includegraphics[width=.8\linewidth]{figures/RRT1.pdf}
}
\end{frame}

\begin{frame} {Rapidly exploring Random Tree (RRT) 2000}
\centerline {
  \includegraphics[width=.8\linewidth]{figures/RRT2.pdf}
}
\end{frame}

\begin{frame} {Rapidly exploring Random Tree (RRT) 2000}
\centerline {
  \includegraphics[width=.8\linewidth]{figures/RRT3.pdf}
}
\end{frame}

\begin{frame} {Rapidly exploring Random Tree (RRT) 2000}
\centerline {
  \includegraphics[width=.8\linewidth]{figures/RRT4.pdf}
}
\end{frame}

\begin{frame} {Rapidly exploring Random Tree (RRT) 2000}
\centerline {
  \includegraphics[width=.8\linewidth]{figures/RRT5.pdf}
}
\end{frame}

\begin{frame} {Rapidly exploring Random Tree (RRT) 2000}
\centerline {
  \includegraphics[width=.8\linewidth]{figures/RRT6.pdf}
}
\end{frame}

\begin{frame} {Rapidly exploring Random Tree (RRT) 2000}
\centerline {
  \includegraphics[width=.8\linewidth]{figures/RRT7.pdf}
}
\end{frame}

\begin{frame} {Rapidly exploring Random Tree (RRT) 2000}
\centerline {
  \includegraphics[width=.8\linewidth]{figures/RRT8.pdf}
}
\end{frame}

\begin{frame} {Rapidly exploring Random Tree (RRT) 2000}
\centerline {
  \includegraphics[width=.8\linewidth]{figures/RRT9.pdf}
}
\end{frame}

\begin{frame} {Rapidly exploring Random Tree (RRT) 2000}
\centerline {
  \includegraphics[width=.8\linewidth]{figures/RRT10.pdf}
}
\end{frame}

\begin{frame} {Rapidly exploring Random Tree (RRT) 2000}
\centerline {
  \includegraphics[width=.8\linewidth]{figures/RRT11.pdf}
}
\end{frame}

\begin{frame} {Rapidly exploring Random Tree (RRT) 2000}
\centerline {
  \includegraphics[width=.8\linewidth]{figures/RRT12.pdf}
}
\end{frame}

\begin{frame} {Rapidly exploring Random Tree (RRT) 2000}
\centerline {
  \includegraphics[width=.8\linewidth]{figures/RRT13.pdf}
}
\end{frame}

\begin{frame} {Rapidly exploring Random Tree (RRT) 2000}
\centerline {
  \includegraphics[width=.8\linewidth]{figures/RRT14.pdf}
}
\end{frame}

\begin{frame} {Rapidly exploring Random Tree (RRT) 2000}
\centerline {
  \includegraphics[width=.8\linewidth]{figures/RRT15.pdf}
}
\end{frame}

\begin{frame} {Rapidly exploring Random Tree (RRT) 2000}
\centerline {
  \includegraphics[width=.8\linewidth]{figures/RRT16.pdf}
}
\end{frame}

\begin{frame} {Rapidly exploring Random Tree (RRT) 2000}
\centerline {
  \includegraphics[width=.8\linewidth]{figures/RRT17.pdf}
}
\end{frame}

\begin{frame} {Rapidly exploring Random Tree (RRT) 2000}
\centerline {
  \includegraphics[width=.8\linewidth]{figures/RRT18.pdf}
}
\end{frame}

\begin{frame} {Rapidly exploring Random Tree (RRT) 2000}
\centerline {
  \includegraphics[width=.8\linewidth]{figures/RRT19.pdf}
}
\end{frame}

\begin{frame} {Rapidly exploring Random Tree (RRT) 2000}
\centerline {
  \includegraphics[width=.8\linewidth]{figures/RRT20.pdf}
}
\end{frame}

%
% random methods
%

\begin{frame} {M\'ethodes al\'eatoires}
  \begin{itemize}
  \item Avantages:
    \begin{itemize}
    \item pas de calcul explicites de l'espace des configurations libres,
    \item facile \'a impl\'ementer,
    \item robuste.
    \end{itemize}
    \pause
  \item Inconv\'enients:
    \begin{itemize}
    \item pas de compl\'etude, seulement une compl\'etude en probabilit\'e.
    \item difficile de trouver un passage \'etroit.
    \end{itemize}
    \pause
  \item Op\'erations requises~:
    \begin{itemize}
    \item test de collisions
      \begin{itemize}
      \item pour des configurations (statique),
      \item pour des chemins (dynamique)
      \end{itemize}
    \end{itemize}
  \end{itemize}
\end{frame}
